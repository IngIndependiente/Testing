%\usepackage[spanish]{babel}
\chapter{Introducción}

El siguiente apartado corresponde al resumen de la información recopilada por cada  software a inspeccionar. El proposito de este documento es presentar una \textit{estrategia de validación} por item, de acuerdo a las funcionalidades a evaluar, arquitectura del sistema y ambiente (Hardware) donde basa sus operaciones.

\chapter{Salida protocolo de riesgo - HPLC agilent 1120}
\begin{itemize}
	\item \textbf{Elemento a evaluar:} Agilent OpenLAB CDS (EZChrom Edition)
	\item \textbf{Versión:} A.04.07
\end{itemize}
\subsubsection{Descripción clara del propósito de la evaluación}
\begin{enumerate}
\item Copia de seguridad de los registros.
\item Protección de los registros.
\item Acceso autorizado a usuario según políticas de acceso.
\end{enumerate}
\subsubsection{Hardware donde opera el elemento}
\begin{itemize}
	\item \textbf{Procesador:} Intel(R) Core(TM) CPU @ 3.10Ghz
	\item \textbf{HDD:} 500GB.
	\item \textbf{RAM:} 8GB
	\item \textbf{Sistema operativo:} Microsoft Windows 7 Professional x64.
	\item \textbf{Sistema de almacenamiento:} Sistema de archivos local.
\end{itemize}
\subsubsection{Principales funcionalidades del elemento a evaluar}
\begin{itemize}
 \item CRUD método, secuencia.
 \item Reproceso de datos.
\end{itemize}
\subsubsection{Manejo de excepciones}
\begin{itemize}
	\item Se muestra un mensaje de error en la interfaz de usuario, cuando se tiene lugar a una falla eléctrica, mecánica, hidráulica o mal uso del sistema. 
\end{itemize}
\subsubsection{Estrategia de validación}
La validación de este sistema consiste en la ejecución de pruebas \textit{funcionales y aceptación}.
\begin{itemize}
\item Pruebas funcionales: Son separadas en \textit{escenarios de uso}, a estos se acoplan una serie de \textit{casos de uso}, es importante que cada escenario sea ejecutado por los encargados de la validación, buscando determinar, si las funciones del sistema se comportan de forma correcta y constante.
\item Prueba de aceptación: Este nivel es ejecutado por los usuarios finales del software con la asistencia de los encargados de la validación. Los usuarios ingresan al sistema según políticas de acceso, haciendo uso de las credenciales pertinentes. Las pruebas de este nivel son con respecto a las necesidades del usuario, requerimientos y uso real del sistema bajo condiciones cotidianas de operación. Se busca determinar si el software cumple con el propósito para el cual fue requerido, con una prueba formal que integre la ejecución de todos los escenarios de uso. Los sistemas ingresan al nivel aceptación una vez que todos los defectos críticos sean corregidos. Un programa puede tener un defecto siempre que este no interfiera con la ejecución de las pruebas, sin embargo la totalidad de los defectos deben ser corregidos antes de finalizar este último nivel.
\end{itemize}
\chapter{Salida protocolo de riesgo - GC Claus 600}
\begin{itemize}
	\item \textbf{Elemento a evaluar:} Perkin Elmer - TotalChrom navigator
	\item \textbf{Versión:} 6.3.2
\end{itemize}
\subsubsection{Descripción clara del propósito de la evaluación}
\begin{enumerate}
	\item Copia de seguridad de los registros.
	\item Protección de los registros.
	\item Acceso autorizado a individuos según políticas de acceso.
\end{enumerate}
\subsubsection{Hardware donde opera el elemento}
\begin{itemize}
	\item \textbf{Procesador:} Intel(R) Core(TM) I5-3470S CPU @ 2.9GHz
	\item \textbf{HDD:} 500GB.
	\item \textbf{RAM:} 8GB
	\item \textbf{Sistena operativo:} Microsoft Windows 7 Professional x64.
	\item \textbf{Sistema de almacenamiento:} Sistema de archivos local.
\end{itemize}
\subsubsection{Principales funcionalidades del elemento a evaluar}
\begin{itemize}
	\item CRUD método, secuencia.
	\item Reproceso de datos.
\end{itemize}
\subsubsection{Manejo de excepciones}
\begin{itemize}
	\item Se muestra un mensaje de error en la interfaz de usuario, cuando se tiene lugar a una falla eléctrica, mecánica o mal uso del sistema. 
\end{itemize}
\subsubsection{Estrategia de validación}
La validación de este sistema consiste en la ejecución de pruebas \textit{funcionales y aceptación}.
\begin{itemize}
	\item Pruebas funcionales: Son separadas en escenarios de uso, a estos se acoplan una serie de casos de prueba, es importante que cada escenario sea ejecutado por los encargados de la validación.
	
	\item Prueba de aceptación: Este nivel es ejecutado por los usuarios finales del software con la asistencia de los encargados de la validación. Los usuarios ingresan al sistema según políticas de acceso, haciendo uso de las credenciales pertinentes. Las pruebas de este nivel son con respecto a las necesidades del usuario, requerimientos y uso real del sistema bajo condiciones cotidianas de operación. Se busca determinar si el software cumple con el propósito para el cual fue requerido con una prueba formal que integre la ejecución de todos los escenarios de prueba. Los sistemas ingresan al nivel aceptación una vez que todos los defectos críticos sean corregidos. Un programa puede tener un defecto siempre que este no interfiera con la ejecución de las pruebas, sin embargo la totalidad de los defectos deben ser corregidos antes de finalizar el último nivel.
\end{itemize}
\chapter{Salida protocolo de riesgo - Bettersize BT-9300H}
\begin{itemize}
	\item \textbf{Elemento a evaluar:} Bettersize laser particle analizer
	\item \textbf{Versión:} 5.0
\end{itemize}
\subsubsection{Descripción clara del propósito de la evaluación}
\begin{enumerate}
	\item Copia de seguridad de los registros.
	\item Protección de los registros.
	\item Acceso autorizado a individuos según políticas de acceso.
\end{enumerate}
\subsubsection{Hardware donde opera el elemento}
\begin{itemize}
	\item \textbf{Procesador:} Intel(R) Core(TM) I5-3470S CPU @ 2.9GHz
	\item \textbf{HDD:} 500GB.
	\item \textbf{RAM:} 8GB
	\item \textbf{Sistena operativo:} Microsoft Windows 7 Professional x64.
	\item \textbf{Sistema de almacenamiento:} Sistema de archivos local.
\end{itemize}
\subsubsection{Principales funcionalidades del elemento a evaluar}
\begin{itemize}
	\item Analizar tamaño de partícula.
\end{itemize}
\subsubsection{Manejo de excepciones}
\begin{itemize}
	\item Se muestra un mensaje de error en la interfaz de usuario, cuando se tiene lugar a una falla eléctrica o mal uso del sistema. 
\end{itemize}
\subsubsection{Estrategia de validación}
La validación de este sistema consiste en la ejecución de pruebas \textit{funcionales y aceptación}.
\begin{itemize}
	\item Pruebas funcionales: Son separadas en escenarios de uso, a estos se acoplan una serie de casos de prueba, es importante que cada escenario sea ejecutado por los encargados de la validación. Se busca determinar si las funciones del sistema se comportan de forma correcta y constante.
	
	\item Prueba de aceptación: Este nivel es ejecutado por los usuarios finales del software con la asistencia de los encargados de la validación. Los usuarios ingresan al sistema según políticas de acceso, haciendo uso de las credenciales pertinentes. Las pruebas de este nivel son con respecto a las necesidades del usuario, requerimientos y uso real del sistema bajo condiciones cotidianas de operación. Se busca determinar si el software cumple con el propósito por el cual fue requerido con una prueba formal que integre la ejecución de todos los escenarios de prueba. Los sistemas ingresan al nivel aceptación una vez que todos los defectos críticos sean corregidos. Un programa puede tener un defecto siempre que este no interfiera con la ejecución de las pruebas, sin embargo la totalidad de los defectos deben ser corregidos antes de finalizar el último nivel.
	
\end{itemize}
\chapter{Salida protocolo de riesgo - Software ISO}
\begin{itemize}
	\item \textbf{Elemento a evaluar:} Software para la gestión documental ISO
\end{itemize}
\subsubsection{Descripción clara del propósito de la evaluación}
\begin{enumerate}
	\item Copia de seguridad de los registros.
	\item Protección de los registros.
	\item Acceso autorizado a individuos según políticas de acceso.
\end{enumerate}
\subsubsection{Hardware donde opera el elemento}
\begin{itemize}
	\item \textbf{Procesador:} Intel(R) Core(TM) I5-3470S CPU @ 2.9GHz
	\item \textbf{HDD:} 500GB.
	\item \textbf{RAM:} 8GB
	\item \textbf{Sistena operativo:} Microsoft Windows 7 Professional x64.
	\item \textbf{Sistema de almacenamiento:} Sistema de archivos local.
\end{itemize}
\subsubsection{Principales funcionalidades del elemento a evaluar}
\begin{itemize}
	\item CRUD Documentos.
	\item CRUD No conformidades.
	\item Flujo de documentos.
\end{itemize}
\subsubsection{Manejo de excepciones}
\begin{itemize}
	\item Se muestra un mensaje de error en la interfaz de usuario cuando ocurre una falla en el sistema
\end{itemize}
\subsubsection{Estrategia de validación}
La validación de este sistema consiste en la ejecución de pruebas \textit{funcionales, sistema y aceptación}.
\begin{itemize}
	\item Pruebas funcionales: Son separadas en escenarios de uso, a estos se acoplan una serie de casos de prueba, es importante que cada escenario sea ejecutado por los encargados de la validación. Se busca determinar si las funciones del sistema se comportan de forma correcta y constante.
	
	\item Pruebas de sistema: Este nivel corresponde a la ejecución de pruebas de estrés haciendo uso de herramientas que simulan el flujo de usuarios, realizando peticiones GET/POST bajo protocolos http y https.
	\item Prueba de aceptación: Este nivel es ejecutado por los usuarios finales del software con la asistencia de los encargados de la validación. Los usuarios ingresan al sistema según políticas de acceso, haciendo uso de las credenciales pertinentes. Las pruebas de este nivel son con respecto a las necesidades del usuario, requerimientos y uso real del sistema bajo condiciones cotidianas de operación. Se busca determinar si el software cumple con el propósito por el cual fue requerido con una prueba formal que integre la ejecución de todos los escenarios de prueba. Los sistemas ingresan al nivel aceptación una vez que todos los defectos críticos sean corregidos. Un programa puede tener un defecto siempre que este no interfiera con la ejecución de las pruebas, sin embargo la totalidad de los defectos deben ser corregidos antes de finalizar el último.
	\end{itemize}